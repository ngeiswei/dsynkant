%%%%%%%%%%%%%%%%%%%%%%%%%%%%%%%%%%%%%%%%%%%%%%%%%%%%%%%%%%%%%%%%%
%%%%%%%%%%%%%%%%% Documentation of DSynkant %%%%%%%%%%%%%%%%%%%%%
%%%%%%%%%%%%%%%%%%%%%%%%%%%%%%%%%%%%%%%%%%%%%%%%%%%%%%%%%%%%%%%%%

\documentclass[11pt]{report}

\title{User Manual of DSynkant}

\author{Nil Geisweiller}

\begin{document}

\maketitle

\tableofcontents

\chapter{Introduction}

\section{What is DSynkant?}

DSynkant is virtual clone of the synthesizer D-50 of Roland.

\section{How to install?}

First you need to install DSSI. then just type :\\
\begin{tabular}{l}
./configure\\
make\\
make install
\end{tabular}

\section{How to use it}

You need a DSSI host, like Rosegarden, MusE, Om\dots
(I advice you not to use jack-dssi-host as it does not accept in input
sysex messages yet). Please note that if you use Rosegarden you won't be able
send patches to DSynkant due to the fact that
Rosegarden loads only the first sysex in a file and disgards the others.
I you use Rosegarden you will have to use an external sysex sender program
like simplesysexxer or MusE.\\

For the moment there is no GUI at all, everything is controlled by MIDI
messages. If you want to edit D-50 patches you can use UniQuest VC-1 (but it
costs money) and D50 Virtual Editor (there is a free version but slightly
imcomplete, although usable). Maybe I will make an editor but it is not
my prior goal.


\end{document}
